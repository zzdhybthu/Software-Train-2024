\documentclass{beamer}
\usepackage{ctex, hyperref}
\usepackage[T1]{fontenc}

% other packages
\usepackage{latexsym,amsmath,xcolor,multicol,booktabs,calligra}
\usepackage{graphicx,pstricks,listings,stackengine}
\usepackage{float}
\usepackage{fontspec}

\usepackage{tsinghua}
% description: classic template for create a titlepage
% defs
\def\cmd#1{\texttt{\color{red}\footnotesize $\backslash$#1}}
\def\env#1{\texttt{\color{blue}\footnotesize #1}}
\definecolor{deepblue}{rgb}{0,0,0.5}
\definecolor{deepred}{rgb}{0.6,0,0}
\definecolor{deepgreen}{rgb}{0,0.5,0}
\definecolor{halfgray}{gray}{0.55}

\lstset{
	basicstyle=\ttfamily\small,
	keywordstyle=\bfseries\color{deepblue},
	emphstyle=\ttfamily\color{deepred},    % Custom highlighting style
	stringstyle=\color{deepgreen},
	numbers=left,
	numberstyle=\small\color{halfgray},
	rulesepcolor=\color{red!20!green!20!blue!20},
	frame=shadowbox,
}
% Title page details: 
\title{2024清华电子\&软院联合暑期培训}
\subtitle{第零讲:MAP}
\author{电子系科协软件部 \& 软件学院科协}
\date{\today}


\begin{document}
	\titlepage
	
	\begin{frame}
        \frametitle{目录}
        \tableofcontents
    \end{frame}

    \section{暑培简介}

    \begin{frame}
        \frametitle{基本信息}
        \begin{itemize}
            \item 暑培对象:电子科协软件部、软院科协新部员,以及全校热爱软件开发的同学
            \item 暑培时间:7.20--8.07,每晚19:30开始,持续 $2\sim3$ 小时
            \item 培训内容:电子科协软件部、软院科协接下来一年中的主要技术栈、渲染技术及拓展讲座
        \end{itemize}
    \end{frame}

    \begin{frame}
        \frametitle{暑培内容}
        分为四大块:通用技术、赛事开发、网站建设、渲染技术、拓展讲座
        \begin{figure}[H]
            \centering
            \includegraphics[width=0.9\linewidth]{yugu.png}
        \end{figure}
    \end{frame}

    \section{通用技术}

    \begin{frame}{开发的基础}
        这一部分中,我们介绍一些软件开发的\textbf{基础知识},无论是软件设计、赛事开发、网站建设乃至未来的科研工作,都离不开这些基础技术的支持。
        \begin{itemize}
            \item \textbf{Day1: Linux基础}:什么是Linux?什么是命令行?什么是Shell?如何使用Linux?
            \item \textbf{Day2:Git与GitHub}:如何使用Git进行版本管理与多人协作?GitHub 怎么用?
            \item \textbf{Day11:Docker}:基础不同的项目运行往往需要不同的环境(依赖包、环境变量、网络端口、以及各种配置),docker 能够帮助我们轻量地搭建、移植环境,避免出现“你电脑可以我电脑不行”的尴尬。
            \item \textbf{Day14:Kotlin开发基础}:深入讲解Kotlin编程语言的基础知识,包括变量、数据类型、控制结构、函数、类与对象等核心概念,展示其简洁高效的语法特点和在现代软件开发中的广泛应用。
        \end{itemize}
    \end{frame}

    \section{赛事开发}

    \begin{frame}{底层逻辑:C\#}
        这一部分中,我们介绍一种现代编程语言C\#。C\#是THUAI赛事开发部分的\textbf{主要语言},从赛事的底层逻辑到外部的游戏界面,都离不开C\#的支持。
        \begin{itemize}
            \item \textbf{Day3: C\#基础}:C\#与C/C++有什么不同?使用C\#有什么好处?C\#基础语法介绍。
            \item \textbf{Day4:多线程}:操作系统如何管理进程与线程?如何使用多线程?多线程有什么好处?多线程中会遇到什么问题?C\#多线程实战。
        \end{itemize}
    \end{frame}

    \begin{frame}{通信与接口}
        这一部分中,我们会先为大家介绍一些关于C++的更多知识,然后介绍Protobuf与gRPC。它们由Google开发,是一种高效的通信协议与通信方式,也是THUAI赛事开发部分的\textbf{核心通信技术}。
        \begin{itemize}
            \item \textbf{Day5: C++进阶}:我们的代码是如何变为可执行程序的?现代的C++有哪些更为高级的特性?C++进阶介绍。
            \item \textbf{Day6:通信基础}:HTTP与TCP/IP协议简介、利用Protobuf和gRPC实现通信、gRPC实战。
        \end{itemize}
    \end{frame}

    \begin{frame}{从命令行到图形界面}
        这一部分中,我们基于C\#,分别介绍WPF/MAUI与Unity/WebGL,它们是THUAI赛事开发部分的\textbf{主要界面技术}。
        \begin{itemize}
            \item \textbf{Day7: WPF/MAUI简介}:使用C\#作为编程语言,利用.NET开发桌面应用程序,WPF/MAUI简介,MVVM思想介绍。
            \item \textbf{Day10:Unity/WebGL简介}:利用Unity制作游戏动画,利用WebGL实现网页上的观看,Unity实战。
        \end{itemize}
    \end{frame}

    \section{网站建设}

    \begin{frame}{前端基础}
        这一部分中,我们会介绍网站建设中使用的一些基础前端技术。
        \begin{itemize}
            \item \textbf{Day12:HTML、CSS}:HTML、CSS是什么?它们在网站开发中起到什么作用?网站开发基础教学。
            \item \textbf{Day13:JavaScript与TypeScript}:利用更为复杂的技术进行网站开发、包管理等操作,网站开发进阶教学。
            \item \textbf{Day17:React与Webpack}:利用现有的React框架,提高网站开发效率,同时使用Webpack进行打包。
        \end{itemize}
    \end{frame}

    \begin{frame}{后端基础}
        这一部分中,我们会介绍网站建设中使用的一些基础后端技术。为了保持技术栈的一致,我们介绍的后端尽可能采取与前端接近的技术栈。
        \begin{itemize}
            \item \textbf{Day15:Database \& SQL \& GraphQL}:介绍关系型数据库 SQL 的基础知识,以及如何使用 GraphQL 进行高效的数据查询与操作。
            \item \textbf{Day16:Node.js与Express}:使用 Node.js 等语言进行后端的开发,使用 Express 将程序打包到服务器上。
            \item \textbf{Day18:COS、CDN、Nginx、CI}:利用简单的Express框架实现web功能、利用COS、CDN等作为云硬盘存储、并介绍Web服务器了解网站部署。
        \end{itemize}
    \end{frame}

    \section{渲染技术}

    \begin{frame}{实时渲染基础魔法}
        这一部分中,我们介绍一些现代实时渲染的基础技术,以及一些现代可导渲染技术。
        \begin{itemize}
            \item \textbf{Day8:实时渲染基础魔法}:从较高的抽象层次出发,介绍现代实时渲染的基础技术,包括可编程光栅渲染管线、物理渲染、非真实感渲染、实时光线追踪技术和后处理技术等。
            \item \textbf{Day9:可导渲染}:将机器学习应用于渲染领域时,会遇到可导性的问题;你是否想过如何对渲染到屏幕上的图形求导?本讲座介绍非物理的现代可导渲染技术,包括神经辐射场(NeRF)、神经有符号距离场(NeuS)和高斯泼溅(Gaussian Splatting)等;以及基于物理的现代可导渲染技术,包括辐射传输的可导理论、路径空间可导渲染和投射采样等。
        \end{itemize}
    \end{frame}

    \section{拓展讲座}

    \begin{frame}{Minecraft红石电路设计}
        \textbf{Day19:Minecraft红石电路设计}:本讲中,我们邀请到了刘浩然学长,为大家介绍 Minecraft 中的红石电路设计。同时,我们将在暑假期间举办红石电路设计大赛,以便大家将\textbf{数字逻辑与处理器基础}课程中所学知识学以致用。
    \end{frame}

    \section{作业与奖品}

    \begin{frame}{作业与奖品}
        \begin{itemize}
            \item \textbf{作业}:每天的课程结束后,我们会布置一些作业,巩固知识,欢迎大家按照讲师的要求提交作业。作业不强制要求完成。
            \item \textbf{奖品}:我们会统计各位同学的作业完成情况,对于完成情况较好的同学,会在下学期初为大家发放奖品。
        \end{itemize}
    \end{frame}
	
	\begin{frame}{}
	    \centering \Huge{\heiti 谢谢大家!}
	\end{frame}
	
\end{document}  